

\section*{第3問}

\subsection*{I. (1). }
\begin{equation}
  \dd{U}=T\dd{S}-p\dd{V}
\end{equation}

\subsection*{I. (2). }
気体は熱力学的に安定なので, $\delta Q>T\delta{S}$であり, また, 系に仮想的変化$\delta U$, $\delta V$を与えると, 
\begin{equation}
  \delta U+p\delta V-T\delta S> 0
\end{equation}
ここで$\delta U$を2次まで展開すると
\begin{equation}
  \begin{multlined}[t]
    \left\{ \left( \pdv{U}{V} \right)_S+p \right\}\delta V+\left\{ \left( \pdv{U}{S} \right)_V-T \right\}\delta S\\
    +\left( \pdv[2]{U}{S} \right)_V(\delta S)^2+2\frac{\partial^2 U}{\partial S \partial V}\delta S\delta V+\left( \pdv[2]{U}{V} \right)_S(\delta V)^2>0
  \end{multlined}
\end{equation}
上式が任意の$\delta V$, $\delta S$について成立するので, 1次の係数は0になるべきで, 
\begin{equation}
  \left( \pdv{U}{V} \right)_S=-p, \quad \left( \pdv{U}{S} \right)_V=T\label{stat_pbI(2)_zenbibun}
\end{equation}
となり, 2次について
\begin{equation}
  \left( \pdv[2]{U}{S} \right)_V(\delta S)^2+2\frac{\partial^2 U}{\partial S \partial V}\delta S\delta V+\left( \pdv[2]{U}{V} \right)_S(\delta V)^2>0\label{stat_pbI(2)}
\end{equation}
を得る\footnote{以上の議論は\cite{Kubo}の3章を参考にした. 内部エネルギーの凸性を要請して, \eqref{stat_pbI(2)}式を導出する方針でも良いと思う. }. 



\subsection*{I. (3). }
\eqref{stat_pbI(2)}式を変形すれば
\begin{equation}
  \begin{pmatrix}
    \delta S&\delta V
  \end{pmatrix}
  \cdot
  \begin{pmatrix}
    \left( \pdv[2]{U}{S} \right)_V&\frac{\partial^2 U}{\partial S \partial V}\\
    \frac{\partial^2 U}{\partial S \partial V}&\left( \pdv[2]{U}{V} \right)_S
  \end{pmatrix}
  \cdot
  \begin{pmatrix}
    \delta S\\
    \delta V
  \end{pmatrix}
  >
  0
\end{equation}
となるので, Hessianを$A$として
\begin{equation}
  A\coloneqq 
    \begin{pmatrix}
    \left( \pdv[2]{U}{S} \right)_V&\frac{\partial^2 U}{\partial S \partial V}\\
    \frac{\partial^2 U}{\partial S \partial V}&\left( \pdv[2]{U}{V} \right)_S
  \end{pmatrix}
\end{equation}
は正定値とわかる. 
従って特に$\text{det}A>0$である. 

以上を踏まえて$C_P$, $\kappa_T$を変形していく. 
\begin{equation}
  C_P=T\left( \pdv{S}{T} \right)_P
\end{equation}
とかけるが, 
\begin{align*}
  \left( \pdv{S}{T} \right)_P
  &=\frac{\partial (S,P)}{\partial (S,V)}\frac{\partial (S,V)}{\partial (T,P)}\\
  &=\left( \pdv{P}{V} \right)_S\frac{1}{\frac{\partial (T,P)}{\partial (S,V)}}\\
  &=-\left( \pdv[2]{U}{V} \right)_S\frac{1}{-\left( \pdv[2]{U}{S} \right)_V\left( \pdv[2]{U}{V} \right)_S+\frac{\partial^2 U}{\partial S\partial V}\frac{\partial^2 U}{\partial S\partial V}}\\
  &=\left( \pdv[2]{U}{V} \right)_S\frac{1}{\text{det}{A}}
\end{align*}
と変形できる. 
途中\eqref{stat_pbI(2)_zenbibun}式を使った. 
\eqref{stat_pbI(2)}式で特に$\delta S=0$とすると
\begin{equation}
  \left( \pdv[2]{U}{V} \right)_S>0
\end{equation}
を得る. 
以上より, 
\begin{equation}
  \left( \pdv{S}{T} \right)_P>0
\end{equation}
であり, $T>0$でもあるので$C_P>0$を得る\footnote{この, 内部エネルギーの自然な変数についてのHessianの任意の主小行列式が非負であることを使った$C_P\geq 0$の証明はYさんに教えてもらった. Jacobianの定義やそれを使った計算方法, Hessianの任意の主小行列式が非負, については\cite{shimizu_thermo_II}を参照. }. 

$\kappa_T$についても同様に
\begin{equation}
  \left( \pdv{V}{P} \right)_T=\frac{\partial (V,T)}{\partial (S,V)}\frac{\partial (S,V)}{\partial (P,T)}=-\left( \pdv[2]{U}{S} \right)_V\frac{1}{\text{det}{A}}<0
\end{equation}
なので, $V>0$と合わせて$\kappa_T>0$が従う\footnote{(別解)$C_P>0$は$C_P-C_V>0$かつ$C_V>0$を使っても解ける(私はこの方針で解いた), $\kappa_T>0$も似た感じで示せたはず. }. 

\subsection*{II. (1). }
例によって$\left( \pdv{T}{P} \right)_H$をJacobianを使って変形していく\footnote{ちなみにこのIIの(1), (2), (3)の第二法則の視点からの解釈以外は\cite{Kubo}第3章A問題[10]とほぼ同じである. }. 
\begin{align*}
  \left( \pdv{T}{P} \right)_H&=\frac{\partial (T,H)}{\partial (P,S)}\frac{\partial (P,S)}{\partial (P,H)}\\
  &=\left( \pdv{T}{P} \right)_S-\left( \pdv{H}{P} \right)_S\left( \pdv{T}{S} \right)_P\left( \pdv{S}{H} \right)_P\\
  &=\left( \pdv{V}{S} \right)_P-V\frac{T}{C_P}\frac{1}{T}\\
  &=\frac{\left( \pdv{V}{T} \right)_P}{\left( \pdv{S}{T} \right)_P}-\frac{V}{C_P}\\
  &=\frac{V}{C_P}\left( \alpha T-1 \right)
\end{align*}
従って
\begin{equation}
  \dd{T}=\frac{V}{C_P}\left( \alpha T-1 \right)\dd{P}
\end{equation}
を得る. 

\subsection*{II. (2). }
今度はエンタルピー一定の過程ではなく, 準静断熱過程, つまりエントロピー一定の過程なので, 
\begin{equation}
  \dd{T}=\left( \pdv{T}{P} \right)_S\dd{P}+\left( \pdv{T}{S} \right)_P\dd{S}\label{stat_pbII_dTtenkai}
\end{equation}
と展開したとき$\dd{S}=0$, よって
\begin{equation}
  \dd{T}=\left( \pdv{T}{P} \right)_S\dd{P}
\end{equation}
とかけ, これを(1)と同じように評価する. 

\begin{align*}
  \left( \pdv{T}{P} \right)_S&=\frac{\partial (T,S)}{\partial (P,T)}\frac{\partial (P,T)}{\partial (P,S)}\\
  &=\left( \pdv{V}{T} \right)_P\frac{T}{C_P}\\
  &=\frac{V\alpha T}{C_P}
\end{align*}
と計算され, 
\begin{equation}
  \dd{T}=\frac{V\alpha T}{C_P}\dd{P}
\end{equation}
を得る. 

\subsection*{II. (3). }
今$\alpha>0$で, $C_P, V, T>0$なので, ${V\alpha T}/{C_P}>V(\alpha T-1)/C_P$であり, $\dd{P}<0$のときは断熱準静操作の方が温度は下がると考えられる. 

また, $\dd{T}$は\eqref{stat_pbII_dTtenkai}のように展開されるが, Joule-Thomson過程は準静ではない断熱過程なので, 熱力学第二法則より$\dd{S}>0$であり, 加えてI.(3)の結果から$\left( \pdv{T}{S} \right)_P>0$でもあるため, 1項目の温度を下げる効果を2項目が打ち消してしまっている. 

\subsection*{III. (1). }

積分を実行する. 

\begin{align*}
  Z&=\begin{multlined}[t]
    \frac{1}{h^{3N}N!}\int\prod_{i=1}^{N}\dd{x_i}\dd{y_i}\dd{z_i}\cdot\int\prod_{i=1}^N\dd{p_{ix}}\dd{p_{iy}}\dd{p_{iz}}e^{-\beta\sum_{i=1}^N\vb{p}^2/(2M)}\\
    \cdot\int\prod_{i=1}^N\dd{p_{i\theta}}e^{-\beta\sum_{i=1}^N p_{i\theta}^2/(2I)}
    \cdot\int\prod_{i=1}^N\dd{\phi_i}
    \cdot \int\prod_{i=1}^N\dd{p_{i\phi}}\dd{\theta_i}e^{-\beta\sum_{i=1}^N\frac{1}{2I}\frac{p_{i\phi}^2}{\sin^2\theta_i}}
  \end{multlined}\\
  &=\frac{V^N}{h^{5N}N!}\left( \int\dd{p}e^{-\beta p^2/(2M)} \right)^{3N}\cdot\left( \int\dd{p_\theta}e^{-\beta p_\theta^2/(2I)} \right)^N\cdot (2\pi)^N\cdot\left( \int\dd{\theta}\int\dd{p_{\phi}}e^{-\beta p_{i\phi}^2/(2I\sin^2\theta_i)} \right)^N\\
  &=\frac{V^N}{h^{5N}N!}\cdot\left( \frac{2I\pi}{\beta} \right)^N\cdot (4\pi)^N\cdot\left( \frac{2M\pi}{\beta} \right)^{3N/2}
\end{align*}

と計算される. 
途中でGauss積分の公式を使った. 

\subsection*{III. (2). }

系のHelmholtz自由エネルギーは$F=-k_B T\log{Z}$で与えられ, 更に熱力学から$P=-\left( \pdv{F}{V} \right)_{T,N}$とかける. 
(1)の計算を代入すれば
\begin{equation}
  P=k_B T\frac{\partial}{\partial V}(N\log V)=\frac{Nk_B T}{V}
\end{equation}
を得る. 

\subsection*{III. (3). }

公式$U=-\frac{\partial}{\partial \beta}(\log{Z})$を使うと
\begin{equation}
  U=-\frac{\partial}{\partial\beta}(-N\log\beta-\frac{3N}{2}\log\beta)=\frac{5Nk_B T}{2}
\end{equation}
従って
\begin{equation}
  C_V=\left( \pdv{U}{T} \right)_V=\frac{5Nk_B}{2}
\end{equation}
を得る. 

\subsection*{III. (4). }
Hamiltonianに付加項$-\sum_{i=1}^N E\mu I\cos\theta_i$が加わるので, 系の分配関数$Z'$は
\begin{align*}
  Z'&=Z\cdot \left( \int_{0}^\pi \dd{\theta}\sqrt{\frac{2I\pi}{\beta}}\sin\theta e^{\beta E\mu\cos\theta} \right)^N/\left( \int\dd{\theta}\int\dd{p_{\phi}}e^{-\beta p_{i\phi}^2/(2I\sin^2\theta_i)} \right)^N\\
  &=Z\cdot \left( \frac{\sinh \beta E\mu}{\beta E\mu} \right)^N
\end{align*}
これを使うと, 
\begin{align*}
  \mathcal{P}_z&=\frac{1}{\beta}\frac{\partial}{\partial E}\left( \log{Z'} \right)\cdot\frac{1}{V}\\
  &=\frac{N\mu}{V}\left( \coth{(\beta E\mu)}-\frac{1}{\beta E\mu} \right)
\end{align*}
を得る\footnote{これはLangevin関数, 問題の書きぶり的に$\beta=1/(k_B T)$に直した方がいいかも. }. 
また, 
\begin{equation}
  \coth{\beta E\mu}\xrightarrow{E\to\infty}1, \quad \frac{1}{E\beta\mu}\xrightarrow{E\to\infty}0
\end{equation}
なので
\begin{equation}
  \mathcal{P}_z\xrightarrow{E\to\infty}\frac{N\mu}{V}
\end{equation}
となる\footnote{$E\to \infty$で全ての双極子が同じ上を向く. }. 

\subsection*{III. (5). }
Langevin関数を$x=0$周りでTaylor展開すると, 
\begin{equation}
  \coth(x)-\frac{1}{x}=\left( \frac{1}{x}+\frac{x}{3}+\symcal{O}(x^3) \right)-\frac{1}{x}=\frac{x}{3}+\symcal{O}(x^3)
\end{equation}
なので, (4)の結果から$x=0$付近での傾きは低温の方が大きく, また$E\to\infty$で同じ$N\mu/V$に近づくことを踏まえると, 答えは図\ref{fig:stat_III(5)}のようになると考えられる\footnote{単調増加はLangevin関数を微分すればわかるが, 上凸かどうかは2回微分してもわからなかった. Langevin関数だから形は既知ということなのか??}. 

\begin{figure}[H]
  \centering
  \includesvg[width=13cm]{graph/2-langevin.svg}
  \caption{(5)解答}\label{fig:stat_III(5)}
\end{figure}


\subsection*{III. (6). }

(4), (5)などから

\begin{equation}
  \mathcal{P}_z=\frac{N\mu}{V}\left( \frac{\beta E\mu}{3}+\symcal{O}((\beta E\mu)^3) \right)
\end{equation}
と書けるので, 
\begin{equation}
  \varepsilon_0\chi=\lim_{E\to 0}\pdv{\mathcal{P}_z}{E}=\frac{N\mu^2}{3Vk_B T}, \quad\therefore \chi=\frac{N\mu^2}{3\varepsilon_0 Vk_B T}
\end{equation}
となる. 
従って, 電気感受率の温度依存性のグラフは反比例になる(図略). 

\subsection*{III. (7). }
(4)で得た分配関数$Z'$から内部エネルギーを計算すると, 
\begin{equation}
  U=-\frac{\partial}{\partial \beta}\left( \log{Z}+N\log{\frac{\sinh(\beta E\mu)}{\beta E\mu}} \right)=\frac{7Nk_B T}{2}-NE\mu\coth(\beta E\mu)
\end{equation}
なので, 系の定積熱容量は
\begin{equation}
  C_V=\left( \pdv{U}{T} \right)_V=\frac{7Nk_B}{2}-\frac{Nk_B (\beta E\mu)^2}{\sinh^2(\beta E\mu)}
\end{equation}
と計算される. 
ここで, 
\begin{equation}
  \lim_{x\to 0}\frac{\sinh x}{x}=1, \quad \lim_{x\to\infty}\frac{x}{\sinh{x}}=0
\end{equation}
なので, 
\begin{equation}
  \lim_{T\to 0}C_V=\frac{7Nk_B}{2}, \quad \lim_{T\to \infty}C_V=\frac{5Nk_B}{2}
\end{equation}
となる. 
この結果は
\begin{itemize}
  \item $T\to\infty$: $E$の影響が熱揺らぎに対して小さく無視されて, (3)の結果に漸近していく. 
  \item $T\to 0$: 双極子がが$z$方向に揃ったところからずれる自由度でも熱を受け取れるようになり, $T\to\infty$よりも大きくなっている\footnote{エネルギー等分配則っぽいけれど, 本当に等分配則由来なのかは分からないので, 言及しないのが無難?}. 
\end{itemize}
と解釈できる. 

以上より, 求めるグラフは以下の図\ref{fig:statIII_(7)}のようになる. 
\begin{figure}[H]
  \centering
  \includesvg[width=13cm]{graph/Hinetsu.svg}
  \caption{(7)の解答}\label{fig:statIII_(7)}
\end{figure}







